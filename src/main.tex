\documentclass[12pt,a4paper]{report}

%AJUSTE DO IDIOMA
\usepackage[utf8]{inputenc} %uso de ACENTOS
\usepackage[portuguese]{babel} % IDIOMA do documento
\usepackage[T1]{fontenc} % hifenização de pacotes acentuados

%PACOTES GERAIS
\usepackage{graphicx} %habilita o uso de FIGURAS
\usepackage[shortlabels]{enumitem}%Permite mudar a forma de enumerar
\usepackage{multirow} % Mesclar células na vertical
\usepackage{geometry} % Controlar as margens da página
\usepackage{amsmath}
\geometry{top=20mm,bottom=20mm,left=15mm,right=15mm}  

%AJUSTE DAS FUNÇÕES TRIGONOMÉTRICAS
\providecommand{\sin}{} \renewcommand{\sin}{\mathrm{sen\,}}
\providecommand{\tan}{} \renewcommand{\tan}{\mathrm{tan\,}}
\providecommand{\cos}{} \renewcommand{\cos}{\mathrm{cos\,}}

%%% Figuras exportadas
\usepackage{pgfplots}
\pgfplotsset{compat=1.15}
\usepackage{mathrsfs}
\usetikzlibrary{arrows}
%\usepackage{figuras}

%%% Dados do cabeçalho
\newcommand{\bimestre}{4$^\circ$ bimestre}
\newcommand{\tema}{Geometria Analítica}
\newcommand{\turma}{$3^\circ$ ano informática}
\newcommand{\disciplina}{Matemática III}

%Cabeçalho
\usepackage{fancyhdr}
\setlength{\headheight}{40pt}
\fancyhead[c]{	\begin{tabular}{rll}
		\multirow{4}{*}{\includegraphics[scale=0.1]{ParelhasVerticalCinza.png}} & IFRN - Campus Parelhas & \bimestre\\ 
		& Professor: Fábio Alvaro Dantas & Tema: \tema \\
		& Disciplina: \disciplina & Turma: \turma\\
		& \multicolumn{2}{l}{Assinatura: \framebox(250,12)} \\
	\end{tabular}
}
\cfoot{}
\begin{document} 
	\pagestyle{fancy}
	\noindent
	\begin{enumerate}[label=\textbf{\arabic*}.]

\item Sejam as retas $r$ e $s$ de equações $y = 2x-3$ e $y=-3x+2$. A tangente do ângulo agudo formado pelas retas é\\
a) 0\hfill
b) 1\hfill
c) $\sqrt{3}$\hfill
d) $\dfrac{\sqrt{3}}{3}$

\item Seja $\alpha$ o ângulo formado por duas retas cujos coeficientes angulares são $-\dfrac{1}{3}$ e $\dfrac{1}{3}$. O valor de $\tan \alpha$ é:\\
a) $\dfrac{3}{4}$\hfill
b) 1\hfill
c) $\dfrac{5}{4}$\hfill
d) $\dfrac{3}{2}$


\item Dada a reta $r:2x-3y+5=0$ e o ponto $P(5,6)$, a distância de $P$ à reta $r$ é\\
a) $\sqrt{91}$\hfill
b) $30\sqrt{13}$\hfill
c) $\dfrac{3\sqrt{91}}{91}$\hfill
d) $\dfrac{3\sqrt{13}}{13}$

\item Dois pontos sobre a reta $y=2$ distam 4 unidades da reta $4x-3y+2=0$. A distância, em unidades, entre as abscissas dos pontos é\\
a) 10\hfill
b) 2\hfill
c) 6\hfill
d) 4

\item A reta de equação $x+2y+c=0$:\\
a) É perpendicular à reta $2x+y+c=0$\\
b) É paralela à reta $2x-4y+c=0$\\
c) Tem distância ao ponto $(-c,1)$ igual a zero\\
d) Forma um ângulo de $\dfrac{\pi}{4}$ rd com a reta $3x+y+c=0$

\end{enumerate}

\end{document} 
